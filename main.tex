\documentclass[11pt]{article}

\usepackage[utf8]{inputenc}
\usepackage[margin=1in]{geometry}
\usepackage{times}
\usepackage{graphicx}
\usepackage{url}
\usepackage{hyperref}
\usepackage{listings}
\usepackage{xcolor}
\usepackage{fancyhdr}
\usepackage{amsmath, amssymb}
\usepackage{caption}
\usepackage{subcaption}
\usepackage{booktabs} % For better table lines
\usepackage{pgfplots} % For plotting graphs
\usepackage{tikz} % For drawing graphics

% PGFPlots compatibility
\pgfplotsset{compat=1.18}

% Header/Footer
\pagestyle{fancy}
\fancyhf{}
\rhead{Computer Security Report}
\lhead{\textit{Your Name}}
\cfoot{\thepage}

% Code Listing
\lstset{
basicstyle=\ttfamily\small,
breaklines=true,
backgroundcolor=\color{gray!10},
frame=single,
captionpos=b
}

\title{\textbf{Trends, Transparency, and Legal Evolution in Electronic Surveillance} \ \large }
\author{Your Name \
Institution / Course \
\texttt{your.email@example.com}
}
\date{\today}

\begin{document}

\maketitle

\begin{abstract}
This report investigates the evolving landscape of electronic surveillance, specifically focusing on wiretap trends, their associated costs, and the impact of technological advancements and legal reforms. We analyze data from official U.S. Courts wiretap reports and carrier transparency disclosures, identifying key trends in authorization rates, target offenses, and the increasing challenge posed by encryption. A critical examination of transparency failures, particularly concerning geofence warrants and inconsistent reporting between government agencies and telecommunication providers, is presented.We deduce underlying causal relationships, highlight contradictions in public data, and infer broader implications for digital privacy and civil liberties.
\end{abstract}

\section{Introduction}
\label{sec:introduction}
Electronic surveillance operations in the United States have undergone substantial transformation over the past two decades. This longitudinal analysis examines wiretap data patterns from 2004 to 2023, tracking how law enforcement surveillance practices have evolved alongside technological developments and shifts in criminal activity. Our investigation draws primarily from annual reports mandated by Title III of the Omnibus Crime Control and Safe Streets Act of 1968 (18 U.S.C. § 2519), which requires detailed congressional reporting of intercepted communications.[cite brudda]


This analysis tracks several key metrics across the study period: authorization rates and judicial approval patterns, cost trends and resource allocation, target offense categories and their evolution, success rates measured by arrests and convictions, and the growing prevalence of encryption as an investigative obstacle. By examining these longitudinal trends, we identify patterns that illuminate how electronic surveillance has adapted to technological change, policy shifts, and evolving criminal landscapes.

The scope of electronic surveillance has expanded considerably beyond traditional "wiretapping" as originally conceived under Title III. Modern operations predominantly target portable devices, encompassing cellular communications, text messaging, and application-based communications.[cite] This evolution reflects both technological advancement and the migration of criminal activity to digital platforms. Our analysis documents how these changes have affected surveillance effectiveness, costs, and other broader implications.

Rather than presenting isolated yearly statistics, this longitudinal approach reveals underlying trends, cyclical patterns, and inflection points that single-year analyses often miss. The data demonstrates how external factors, from technological developments to policy changes, have shaped surveillance practices over time, providing insights into both current operations and future trajectories.





\end{document}
